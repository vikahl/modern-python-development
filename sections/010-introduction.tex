{
\metroset{sectionpage=none}
\section*{Introduktion}
}

\begin{frame}{Python 2 EOL}
  \begin{columns}
    \begin{column}{0.7\linewidth}
      \begin{itemize}
        \item Python 2 EOL om 1 månad och 22 dagar
        \item \textbf{Inga} uppdateringar efter det
        \item Migrera nu!
      \end{itemize}
    \end{column}
    \begin{column}{0.3\linewidth}
      \begin{figure}
        \includegraphics[width=\linewidth,keepaspectratio]{fig/python2eol}
        \caption{Lisa Roach, Twitter: @lisroach}
      \end{figure}
    \end{column}
  \end{columns}
\end{frame}


\begin{frame}{Förbehåll}
  \begin{itemize}
    \item "Moderna" Python-versionerna
      \begin{itemize}
        \item Python 3.6 -- 2016-12-23 (säkerhetsuppdateringar t.o.m. 2021)
        \item Python 3.7 -- 2018-06-27
        \item Python 3.8 -- 2019-10-14
      \end{itemize}
    \item Exempel fungerar på Linux/Mac, oftast också på Windows
  \end{itemize}
\end{frame}


\begin{frame}{Agenda}
  \setcounter{tocdepth}{1}
  \tableofcontents
\end{frame}

\begin{frame}{todo-extractor}
  \begin{itemize}
    \item Paket som kommer användas som exempel under föreläsningen
    \item Extraherar \textit{TODO} från kod och retunerar JSON
    \item Bra som exempelmodul, men begränsad "riktig" användbarhet
  \end{itemize}

  \faicon{github}/vikahl/todo-extractor

  \gittag{first-version}
\end{frame}
